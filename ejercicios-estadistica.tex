% Options for packages loaded elsewhere
\PassOptionsToPackage{unicode}{hyperref}
\PassOptionsToPackage{hyphens}{url}
\PassOptionsToPackage{dvipsnames,svgnames,x11names}{xcolor}
%
\documentclass[
  a4paper,
]{scrreport}

\usepackage{amsmath,amssymb}
\usepackage{iftex}
\ifPDFTeX
  \usepackage[T1]{fontenc}
  \usepackage[utf8]{inputenc}
  \usepackage{textcomp} % provide euro and other symbols
\else % if luatex or xetex
  \usepackage{unicode-math}
  \defaultfontfeatures{Scale=MatchLowercase}
  \defaultfontfeatures[\rmfamily]{Ligatures=TeX,Scale=1}
\fi
\usepackage{lmodern}
\ifPDFTeX\else  
    % xetex/luatex font selection
\fi
% Use upquote if available, for straight quotes in verbatim environments
\IfFileExists{upquote.sty}{\usepackage{upquote}}{}
\IfFileExists{microtype.sty}{% use microtype if available
  \usepackage[]{microtype}
  \UseMicrotypeSet[protrusion]{basicmath} % disable protrusion for tt fonts
}{}
\makeatletter
\@ifundefined{KOMAClassName}{% if non-KOMA class
  \IfFileExists{parskip.sty}{%
    \usepackage{parskip}
  }{% else
    \setlength{\parindent}{0pt}
    \setlength{\parskip}{6pt plus 2pt minus 1pt}}
}{% if KOMA class
  \KOMAoptions{parskip=half}}
\makeatother
\usepackage{xcolor}
\setlength{\emergencystretch}{3em} % prevent overfull lines
\setcounter{secnumdepth}{5}
% Make \paragraph and \subparagraph free-standing
\ifx\paragraph\undefined\else
  \let\oldparagraph\paragraph
  \renewcommand{\paragraph}[1]{\oldparagraph{#1}\mbox{}}
\fi
\ifx\subparagraph\undefined\else
  \let\oldsubparagraph\subparagraph
  \renewcommand{\subparagraph}[1]{\oldsubparagraph{#1}\mbox{}}
\fi


\providecommand{\tightlist}{%
  \setlength{\itemsep}{0pt}\setlength{\parskip}{0pt}}\usepackage{longtable,booktabs,array}
\usepackage{calc} % for calculating minipage widths
% Correct order of tables after \paragraph or \subparagraph
\usepackage{etoolbox}
\makeatletter
\patchcmd\longtable{\par}{\if@noskipsec\mbox{}\fi\par}{}{}
\makeatother
% Allow footnotes in longtable head/foot
\IfFileExists{footnotehyper.sty}{\usepackage{footnotehyper}}{\usepackage{footnote}}
\makesavenoteenv{longtable}
\usepackage{graphicx}
\makeatletter
\def\maxwidth{\ifdim\Gin@nat@width>\linewidth\linewidth\else\Gin@nat@width\fi}
\def\maxheight{\ifdim\Gin@nat@height>\textheight\textheight\else\Gin@nat@height\fi}
\makeatother
% Scale images if necessary, so that they will not overflow the page
% margins by default, and it is still possible to overwrite the defaults
% using explicit options in \includegraphics[width, height, ...]{}
\setkeys{Gin}{width=\maxwidth,height=\maxheight,keepaspectratio}
% Set default figure placement to htbp
\makeatletter
\def\fps@figure{htbp}
\makeatother

%\newfontfamily\Ubuntu[Mapping=tex-text]{Ubuntu}
\usepackage{pgfplots}
\usetikzlibrary{arrows.meta,arrows}
\usetikzlibrary{angles,quotes}
\pgfplotsset{grid style={dashed,mygray}}
% Colors
\definecolor{myblue}{rgb}{0.067,0.529,0.871}
\definecolor{mypurple}{rgb}{0.859,0.071,0.525}
\definecolor{myred}{rgb}{1.0, 0.13, 0.32}
\definecolor{mygreen}{rgb}{0.01, 0.75, 0.24}
\definecolor{myblack}{gray}{0.1}
\definecolor{mygray}{gray}{0.8}
\newcommand{\NN}{\mathbb{N}}
\newcommand{\ZZ}{\mathbb{Z}}
\newcommand{\QQ}{\mathbb{Q}}
\newcommand{\RR}{\mathbb{R}}
\newcommand{\CC}{\mathbb{C}}
\DeclareMathOperator{\operatorname{Int}}{Int}
\DeclareMathOperator{\operatorname{Ext}}{Ext}
\DeclareMathOperator{\operatorname{Fr}}{Fr}
\DeclareMathOperator{\Adh}{Adh}
\DeclareMathOperator{\Ac}{Ac}
\DeclareMathOperator{\sen}{sen}
\makeatletter
\@ifpackageloaded{tcolorbox}{}{\usepackage[skins,breakable]{tcolorbox}}
\@ifpackageloaded{fontawesome5}{}{\usepackage{fontawesome5}}
\definecolor{quarto-callout-color}{HTML}{909090}
\definecolor{quarto-callout-note-color}{HTML}{0758E5}
\definecolor{quarto-callout-important-color}{HTML}{CC1914}
\definecolor{quarto-callout-warning-color}{HTML}{EB9113}
\definecolor{quarto-callout-tip-color}{HTML}{00A047}
\definecolor{quarto-callout-caution-color}{HTML}{FC5300}
\definecolor{quarto-callout-color-frame}{HTML}{acacac}
\definecolor{quarto-callout-note-color-frame}{HTML}{4582ec}
\definecolor{quarto-callout-important-color-frame}{HTML}{d9534f}
\definecolor{quarto-callout-warning-color-frame}{HTML}{f0ad4e}
\definecolor{quarto-callout-tip-color-frame}{HTML}{02b875}
\definecolor{quarto-callout-caution-color-frame}{HTML}{fd7e14}
\makeatother
\makeatletter
\@ifpackageloaded{bookmark}{}{\usepackage{bookmark}}
\makeatother
\makeatletter
\@ifpackageloaded{caption}{}{\usepackage{caption}}
\AtBeginDocument{%
\ifdefined\contentsname
  \renewcommand*\contentsname{Tabla de contenidos}
\else
  \newcommand\contentsname{Tabla de contenidos}
\fi
\ifdefined\listfigurename
  \renewcommand*\listfigurename{Listado de Figuras}
\else
  \newcommand\listfigurename{Listado de Figuras}
\fi
\ifdefined\listtablename
  \renewcommand*\listtablename{Listado de Tablas}
\else
  \newcommand\listtablename{Listado de Tablas}
\fi
\ifdefined\figurename
  \renewcommand*\figurename{Figura}
\else
  \newcommand\figurename{Figura}
\fi
\ifdefined\tablename
  \renewcommand*\tablename{Tabla}
\else
  \newcommand\tablename{Tabla}
\fi
}
\@ifpackageloaded{float}{}{\usepackage{float}}
\floatstyle{ruled}
\@ifundefined{c@chapter}{\newfloat{codelisting}{h}{lop}}{\newfloat{codelisting}{h}{lop}[chapter]}
\floatname{codelisting}{Listado}
\newcommand*\listoflistings{\listof{codelisting}{Listado de Listados}}
\usepackage{amsthm}
\theoremstyle{definition}
\newtheorem{exercise}{Ejercicio}[chapter]
\theoremstyle{remark}
\AtBeginDocument{\renewcommand*{\proofname}{Prueba}}
\newtheorem*{remark}{Observación}
\newtheorem*{solution}{Solución}
\newtheorem{refremark}{Observación}[chapter]
\newtheorem{refsolution}{Solución}[chapter]
\makeatother
\makeatletter
\makeatother
\makeatletter
\@ifpackageloaded{caption}{}{\usepackage{caption}}
\@ifpackageloaded{subcaption}{}{\usepackage{subcaption}}
\makeatother
\ifLuaTeX
\usepackage[bidi=basic]{babel}
\else
\usepackage[bidi=default]{babel}
\fi
\babelprovide[main,import]{spanish}
% get rid of language-specific shorthands (see #6817):
\let\LanguageShortHands\languageshorthands
\def\languageshorthands#1{}
\ifLuaTeX
  \usepackage{selnolig}  % disable illegal ligatures
\fi
\usepackage{bookmark}

\IfFileExists{xurl.sty}{\usepackage{xurl}}{} % add URL line breaks if available
\urlstyle{same} % disable monospaced font for URLs
\hypersetup{
  pdftitle={Problemas de Estadística},
  pdfauthor={Alfredo Sánchez Alberca},
  pdflang={es},
  colorlinks=true,
  linkcolor={blue},
  filecolor={Maroon},
  citecolor={Blue},
  urlcolor={Blue},
  pdfcreator={LaTeX via pandoc}}

\title{Problemas de Estadística}
\author{Alfredo Sánchez Alberca}
\date{2022-01-06}

\begin{document}
\begin{titlepage}

%\AddToShipoutPicture*{\put(0,0){\includegraphics[scale=0.8]{img/background2}}} % Imagen de fondo, requiere el paquete eso-pic.
\begin{center}
\vspace*{5cm}

\Huge
{\textbf{\textsf{Problemas de Estadística}}}

\vspace{0.5cm}
\LARGE
{\textbf{\textsf{}}}

\vspace{1.5cm}

\includegraphics[width=0.4\textwidth]{img/logos/sticker.png}
\end{center}

\vfill

\begin{flushleft}
\begin{tabular}{ll}
\includegraphics[width=0.1\textwidth]{img/logos/aprendeconalf.png} & \parbox[b]{5cm}{\Large\textsf{Alfredo
Sánchez
Alberca}\\ \textsf{asalber@ceu.es} \\ \textsf{https://aprendeconalf.es}}
\end{tabular}
\end{flushleft}
\end{titlepage}
\renewcommand*\contentsname{Tabla de contenidos}
{
\hypersetup{linkcolor=}
\setcounter{tocdepth}{2}
\tableofcontents
}
\bookmarksetup{startatroot}

\chapter*{Prefacio}\label{prefacio}
\addcontentsline{toc}{chapter}{Prefacio}

\markboth{Prefacio}{Prefacio}

Colección de problemas de Estadística aplicada a la Economía para el
Master en Análisis y Comunicación de Datos.

\bookmarksetup{startatroot}

\chapter{Estimación de parámetros}\label{estimaciuxf3n-de-paruxe1metros}

\begin{exercise}[]\protect\hypertarget{exr-distribución-media-trabajadores-pymes}{}\label{exr-distribución-media-trabajadores-pymes}

El número medio de trabajadores en las PYMES españolas es \(5\) y su
varianza \(4\). Realizado un muestreo aleatorio de \(16\) PYMES,
calcular:

\begin{enumerate}
\def\labelenumi{\alph{enumi}.}
\item
  La esperanza y varianza de la media muestral.
\item
  La esperanza de la varianza y de la cuasivarianza muestral.
\item
  Mínimo tamaño que ha de tener la muestra para que exista una
  probabilidad mayor o igual al \(95\)\% de que la media muestral se
  desvíe de la media poblacional a lo sumo \(0.5\) unidades.
\item
  Si realizamos un muestreo aleatorio de tamaño \(320\) obtener
  \(P(4.9\leq \bar x  \leq 5.2)\).
\end{enumerate}

\end{exercise}

\begin{tcolorbox}[enhanced jigsaw, opacitybacktitle=0.6, rightrule=.15mm, toprule=.15mm, arc=.35mm, left=2mm, toptitle=1mm, title=\textcolor{quarto-callout-tip-color}{\faLightbulb}\hspace{0.5em}{Solución}, titlerule=0mm, opacityback=0, colback=white, bottomtitle=1mm, bottomrule=.15mm, colframe=quarto-callout-tip-color-frame, leftrule=.75mm, breakable, coltitle=black, colbacktitle=quarto-callout-tip-color!10!white]

Sea \(X\) la variable aleatoria que mide el número de trabajadores en
una muestra de 16 PYMES españolas.

\begin{enumerate}
\def\labelenumi{\alph{enumi}.}
\tightlist
\item
  \(E(\bar x) = 5\) y \(Var(\bar x) = \frac{4}{16} = 0.25\).
\item
  \(E(s^2) = 3.75\) y \(E(\hat s^2) = 4\).
\item
  \(P(|\bar x - 5| \leq 0.5) = 0.95\) si \(n\geq 320\).
\item
  Sea \(Y\) la variable aleatoria que mide el número de trabajadores en
  una muestra de 320 PYMES españolas. Entonces,
  \(Y\sim N\left(5, \sqrt{\frac{4}{320}}\right)\) y
  \(P(4.9\leq \bar x  \leq 5.2) = 0.777\).
\end{enumerate}

\end{tcolorbox}

\begin{exercise}[]\protect\hypertarget{exr-distribución-cuasivarianza-iberpapel}{}\label{exr-distribución-cuasivarianza-iberpapel}

El precio de las acciones de Iberpapel se distribuyen según un modelo
normal \(N(\mu, 2)\). Si se analizan \(16\) sesiones de la Bolsa de
Madrid elegidas aleatoriamente, ¿cuál es la probabilidad de que la
cuasivarianza muestral del precio de las acciones sea mayor o igual que
\(2.136\)?

\end{exercise}

\begin{tcolorbox}[enhanced jigsaw, opacitybacktitle=0.6, rightrule=.15mm, toprule=.15mm, arc=.35mm, left=2mm, toptitle=1mm, title=\textcolor{quarto-callout-tip-color}{\faLightbulb}\hspace{0.5em}{Solución}, titlerule=0mm, opacityback=0, colback=white, bottomtitle=1mm, bottomrule=.15mm, colframe=quarto-callout-tip-color-frame, leftrule=.75mm, breakable, coltitle=black, colbacktitle=quarto-callout-tip-color!10!white]

Sabemos que \(\frac{ns^2}{\sigma^2}\sim \chi^2(n-1)\). Por tanto,
\(P(\hat s^2\geq 2.136) = P\left(\frac{16\cdot s^2}{2}\geq 4\cdot 2.136\right) = P\left(\chi^2(15)\geq 8.544\right) = 0.9\).

\end{tcolorbox}

\begin{exercise}[]\protect\hypertarget{exr-diferencia-proporciones-votos}{}\label{exr-diferencia-proporciones-votos}

El porcentaje de votantes con preferencia de un determinado partido es
del \(5\)\% en una región \(A\), y el \(10\)\% en otra \(B\).
Consultados \(100\) electores de la región \(A\) y \(150\) de la \(B\),
determinar la probabilidad de que el porcentaje de electores consultados
favorables a dicho partido en la segunda región supere en más de \(2\)\%
al porcentaje de electores favorables a dicho partido en la primera.

\end{exercise}

\begin{tcolorbox}[enhanced jigsaw, opacitybacktitle=0.6, rightrule=.15mm, toprule=.15mm, arc=.35mm, left=2mm, toptitle=1mm, title=\textcolor{quarto-callout-tip-color}{\faLightbulb}\hspace{0.5em}{Solución}, titlerule=0mm, opacityback=0, colback=white, bottomtitle=1mm, bottomrule=.15mm, colframe=quarto-callout-tip-color-frame, leftrule=.75mm, breakable, coltitle=black, colbacktitle=quarto-callout-tip-color!10!white]

Sea \(X_1\) proporción de votantes favorables al partido en la región
\(A\) en una muestra de \(100\) electores y \(X_2\) proporción de
votantes favorables al partido en la región \(B\) en una muestra de
\(150\) electores. Entonces,
\(X_1\sim N\left(0.05, \sqrt{\frac{0.05\cdot (1-0.05)}{100}}\right)\) y
\(X_2\sim N\left(0.1, \sqrt{\frac{0.1\cdot (1-0.1)}{150}}\right)\), y
\(P(X_2-X_1>0.02) = 0.8212\).

\end{tcolorbox}

\begin{exercise}[]\protect\hypertarget{exr-intervalo-confianza-media-carne-porcino}{}\label{exr-intervalo-confianza-media-carne-porcino}

Se sabe que el gasto mensual en carne de porcino en las familias
españolas se distribuye de manera normal. Se ha realizado un muestreo
aleatorio simple en el que se ha preguntado a \(20\) familias sobre el
gasto mensual en carne de porcino, y se ha obtenido una media de
\(170.31\) € y una cuasidesviación típica de \(36\) €.

\begin{enumerate}
\def\labelenumi{\alph{enumi}.}
\item
  Obtener el intervalo de confianza para el gasto medio mensual en carne
  de porcino con un 95\% de confianza.
\item
  ¿Cómo podría obtenerse un intervalo de confianza más preciso para el
  gasto medio suponiendo que no varían la media muestral, la
  cuasivarianza muestral y el tamaño de la muestra? ¿Y si no varían la
  media muestral, la cuasivarianza muestral y el nivel de confianza?
\item
  Obtener el intervalo de confianza para la desviación típica del gasto
  mensual en carne de porcino con un 95\% de confianza.
\end{enumerate}

\end{exercise}

\begin{tcolorbox}[enhanced jigsaw, opacitybacktitle=0.6, rightrule=.15mm, toprule=.15mm, arc=.35mm, left=2mm, toptitle=1mm, title=\textcolor{quarto-callout-tip-color}{\faLightbulb}\hspace{0.5em}{Solución}, titlerule=0mm, opacityback=0, colback=white, bottomtitle=1mm, bottomrule=.15mm, colframe=quarto-callout-tip-color-frame, leftrule=.75mm, breakable, coltitle=black, colbacktitle=quarto-callout-tip-color!10!white]

Sea \(X\sim N(\mu,\sigma)\) la variable aleatoria que mide el gasto
mensual en carne de porcino en una muestra de 20 familias españolas.

\begin{enumerate}
\def\labelenumi{\alph{enumi}.}
\item
  Usando el intervalo de confianza de la t de Student se tiene
  \(\mu \in (153.46, 187.16)\) con una confianza del \(95\)\%.
\item
  Para obtener un intervalo de confianza más preciso para el gasto medio
  manteniendo la misma media muestral, cuasivarianza muestral y tamaño
  de la muestra, se puede reducir el nivel de confianza.

  Y para obtener un intervalo de confianza más preciso para el gasto
  medio manteniendo la misma media muestral, cuasivarianza muestral y
  nivel de confianza, se puede aumentar el tamaño de la muestra.
\item
  Usando el intervalo de confianza de la \(\chi^2\) se tiene
  \(\sigma \in (27.37, 52.58)\) con una confianza del \(95\)\%.
\end{enumerate}

\end{tcolorbox}

\begin{exercise}[]\protect\hypertarget{exr-intervalo-confianza-media-peso}{}\label{exr-intervalo-confianza-media-peso}

La OMS ha obtenido una muestra de los pesos de \(50\) niños de \(12\)
años, que proporciona una media muestral de \(47\) kg y una
cuasidesviación típica muestral de \(11\) kg. Suponiendo que la
población sigue una distribución normal:

\begin{enumerate}
\def\labelenumi{\alph{enumi}.}
\item
  Obtener un intervalo de confianza para la media poblacional con un
  \(95\)\% de nivel de confianza.
\item
  El director de la OMS considera que el intervalo es poco preciso, pero
  quiere mantener el nivel de confianza. Por ello decide reducir a la
  mitad la amplitud del intervalo. En estas condiciones, ¿cuál debería
  ser el tamaño de la muestra para cumplir los objetivos del director?
\item
  Los resultados obtenidos en los análisis anteriores siguen sin
  convencer al director de la OMS y le pide a su equipo que establezca
  un intervalo de confianza para la media poblacional con un \(99\)\% de
  nivel de confianza, manteniendo la misma muestra del primer apartado.
\item
  El director decide reducir en un tercio la amplitud del intervalo
  anterior, pero quiere mantener el nivel de confianza ¿cuál debería ser
  el tamaño de la muestra para cumplir dicho objetivo?
\end{enumerate}

\end{exercise}

\begin{tcolorbox}[enhanced jigsaw, opacitybacktitle=0.6, rightrule=.15mm, toprule=.15mm, arc=.35mm, left=2mm, toptitle=1mm, title=\textcolor{quarto-callout-tip-color}{\faLightbulb}\hspace{0.5em}{Solución}, titlerule=0mm, opacityback=0, colback=white, bottomtitle=1mm, bottomrule=.15mm, colframe=quarto-callout-tip-color-frame, leftrule=.75mm, breakable, coltitle=black, colbacktitle=quarto-callout-tip-color!10!white]

Sea \(X\sim N(\mu,\sigma)\) la variable aleatoria que mide el peso de
los niños de 12 años.

\begin{enumerate}
\def\labelenumi{\alph{enumi}.}
\item
  Usando el intervalo de confianza de la t de Student se tiene
  \(\mu \in (43.88, 50.12)\) con una confianza del \(95\)\%.
\item
  Para reducir a la mitad la amplitud del intervalo manteniendo el nivel
  de confianza, se debe cuadruplicar el tamaño de la muestra, es decir,
  se necesita \(n=200\) niños.
\item
  Usando el intervalo de confianza de la t de Student se tiene
  \(\mu \in (42.83, 51.16)\) con una confianza del \(99\)\%.
\item
  Para reducir en un tercio la amplitud del intervalo manteniendo el
  nivel de confianza, se debe multiplicar por \(9\) el tamaño de la
  muestra, es decir, se necesita \(n=450\) niños.
\end{enumerate}

\end{tcolorbox}

\begin{exercise}[]\protect\hypertarget{exr-intervalo-confianza-media-ventas-bicicletas}{}\label{exr-intervalo-confianza-media-ventas-bicicletas}

Un fabricante de bicicletas quiere estimar la media de ventas de
bicicletas en un año. Para ello, ha tomado una muestra aleatoria simple
de \(17\) establecimientos, y ha obtenido una media muestral \(3650\)
bicicletas con una cuasidesviación típica muestral de \(55\) bicicletas.
Suponiendo que las ventas de bicicletas siguen una distribución normal:

\begin{enumerate}
\def\labelenumi{\alph{enumi}.}
\item
  Calcular el intervalo de confianza para la media con un nivel de
  confianza del 95\%.
\item
  Calcular el intervalo de confianza para la varianza con un grado de
  confianza del 95\%.
\end{enumerate}

\end{exercise}

\begin{tcolorbox}[enhanced jigsaw, opacitybacktitle=0.6, rightrule=.15mm, toprule=.15mm, arc=.35mm, left=2mm, toptitle=1mm, title=\textcolor{quarto-callout-tip-color}{\faLightbulb}\hspace{0.5em}{Solución}, titlerule=0mm, opacityback=0, colback=white, bottomtitle=1mm, bottomrule=.15mm, colframe=quarto-callout-tip-color-frame, leftrule=.75mm, breakable, coltitle=black, colbacktitle=quarto-callout-tip-color!10!white]

Sea \(X\sim N(\mu,\sigma)\) la variable aleatoria que mide las ventas de
bicicletas en un año.

\begin{enumerate}
\def\labelenumi{\alph{enumi}.}
\item
  Usando el intervalo de confianza de la t de Student se tiene
  \(\mu \in (3621.70, 3678.30)\) con una confianza del \(95\)\%.
\item
  Usando el intervalo de confianza de la \(\chi^2\) se tiene
  \(\sigma^2 \in (1512.5, 7006.38)\) con una confianza del \(95\)\%.
\end{enumerate}

\end{tcolorbox}

\begin{exercise}[]\protect\hypertarget{exr-intervalo-confianza-media-bitcoin}{}\label{exr-intervalo-confianza-media-bitcoin}

Un banco quiere saber el nivel de implantación de la criptomoneda
Bitcoin y para ello se ha realizado un muestreo aleatorio simple de
\(100\) españoles, resultando que \(15\) tienen bitcoins.

\begin{enumerate}
\def\labelenumi{\alph{enumi}.}
\item
  Obtener un intervalo de confianza del \(95\)\% para la proporción
  poblacional de españoles que poseen bitcoins.
\item
  ¿A cuántos españoles se debería encuestar para lograr una semiamplitud
  del intervalo de \(0.02\), utilizando un nivel de confianza del
  \(90\)\%?
\end{enumerate}

\end{exercise}

\begin{tcolorbox}[enhanced jigsaw, opacitybacktitle=0.6, rightrule=.15mm, toprule=.15mm, arc=.35mm, left=2mm, toptitle=1mm, title=\textcolor{quarto-callout-tip-color}{\faLightbulb}\hspace{0.5em}{Solución}, titlerule=0mm, opacityback=0, colback=white, bottomtitle=1mm, bottomrule=.15mm, colframe=quarto-callout-tip-color-frame, leftrule=.75mm, breakable, coltitle=black, colbacktitle=quarto-callout-tip-color!10!white]

Sea \(X\sim B(100, p)\) la variable aleatoria que mide el número de
españoles que poseen bitcoins.

\begin{enumerate}
\def\labelenumi{\alph{enumi}.}
\item
  Usando el intervalo de confianza de la normal se tiene
  \(p \in (0.0801, 0.2199)\) con una confianza del \(95\)\%.
\item
  Para lograr una semiamplitud del intervalo de \(0.02\) con un nivel de
  confianza del \(90\)\%, se necesita encuestar a \(n=861\) españoles.
\end{enumerate}

\end{tcolorbox}

\begin{exercise}[]\protect\hypertarget{exr-intervalo-confianza-proporción-covid}{}\label{exr-intervalo-confianza-proporción-covid}

Para conocer la prevalencia de la COVID en una población se ha tomado
una muestra aleatoria de \(500\) personas y se ha observado que \(36\)
tenían COVID. ¿Qué precisión tiene el intervalo de confianza del
\(95\)\% para la proporción de personas infectadas en la población? ¿Qué
tamaño muestral habría que tomar para doblar la precisión del intervalo?

\end{exercise}

\begin{tcolorbox}[enhanced jigsaw, opacitybacktitle=0.6, rightrule=.15mm, toprule=.15mm, arc=.35mm, left=2mm, toptitle=1mm, title=\textcolor{quarto-callout-tip-color}{\faLightbulb}\hspace{0.5em}{Solución}, titlerule=0mm, opacityback=0, colback=white, bottomtitle=1mm, bottomrule=.15mm, colframe=quarto-callout-tip-color-frame, leftrule=.75mm, breakable, coltitle=black, colbacktitle=quarto-callout-tip-color!10!white]

Sea \(X\sim B(500, p)\) la variable aleatoria que mide el número de
personas infectadas por COVID.

Usando el intervalo de confianza de la normal se tiene el error en la
estimación es \(E = 0.0227\), es decir, un \(2.27\)\%.

Para doblar la precisión del intervalo se necesita un tamaño muestral de
\(n=1993\) personas.

\end{tcolorbox}

\begin{exercise}[]\protect\hypertarget{exr-intervalo-confianza-proporción-uso-alta-velocidad}{}\label{exr-intervalo-confianza-proporción-uso-alta-velocidad}

Tras la liberalización del transporte ferroviario de pasajeros en las
líneas de alta velocidad en España, la compañía francesa SNCF estudia la
proporción de clientes que utiliza al menos una vez al mes el servicio
de alta velocidad. A tal efecto la empresa realiza un muestreo aleatorio
en el que se seleccionan \(50\) usuarios y en el que resulta que \(35\)
de ellos afirma utilizar este servicio una vez al mes como mínimo.
Calcular el intervalo de confianza del \(98\)\% para la proporción
poblacional de usuarios que utilizan la alta velocidad al menos una vez
al mes.

\end{exercise}

\begin{tcolorbox}[enhanced jigsaw, opacitybacktitle=0.6, rightrule=.15mm, toprule=.15mm, arc=.35mm, left=2mm, toptitle=1mm, title=\textcolor{quarto-callout-tip-color}{\faLightbulb}\hspace{0.5em}{Solución}, titlerule=0mm, opacityback=0, colback=white, bottomtitle=1mm, bottomrule=.15mm, colframe=quarto-callout-tip-color-frame, leftrule=.75mm, breakable, coltitle=black, colbacktitle=quarto-callout-tip-color!10!white]

Sea \(X\sim B(50, p)\) la variable aleatoria que mide el número de
usuarios que utilizan la alta velocidad una vez al mes como mínimo.

Usando el intervalo de confianza de la normal se tiene
\(p \in (0.549, 0.851)\) con una confianza del \(98\)\%.

\end{tcolorbox}

\begin{exercise}[]\protect\hypertarget{exr-tamaño-muestra-proporción-encuesta}{}\label{exr-tamaño-muestra-proporción-encuesta}

Leemos en un periódico que la intención de voto a un partido político
está entre el \(25\)\% y el \(31\)\% con un \(95\)\% de confianza. ¿Cuál
es el tamaño muestral que se ha utilizado para dar esta estimación?

\end{exercise}

\begin{tcolorbox}[enhanced jigsaw, opacitybacktitle=0.6, rightrule=.15mm, toprule=.15mm, arc=.35mm, left=2mm, toptitle=1mm, title=\textcolor{quarto-callout-tip-color}{\faLightbulb}\hspace{0.5em}{Solución}, titlerule=0mm, opacityback=0, colback=white, bottomtitle=1mm, bottomrule=.15mm, colframe=quarto-callout-tip-color-frame, leftrule=.75mm, breakable, coltitle=black, colbacktitle=quarto-callout-tip-color!10!white]

Usando el intervalo de confianza de la normal se tiene que el tamaño
muestral necesario para obtener este intervalo para la proporción de
personas que votaría al partido es \(n=861\).

\end{tcolorbox}

\begin{exercise}[]\protect\hypertarget{exr-intervalo-confianza-comparación-proporciones-voto}{}\label{exr-intervalo-confianza-comparación-proporciones-voto}

En una encuesta realizada a \(1000\) personas sobre la intención de voto
en unas elecciones, \(350\) comentaron que votarían al partido \(A\) y
\(390\) al partido \(B\). Calcular los intervalos de confianza del
\(95\)\% para el porcentaje de voto a cada partido. ¿Se puede afirmar
con un \(95\)\% de confianza que el partido \(B\) ganaría las
elecciones?

\end{exercise}

\begin{tcolorbox}[enhanced jigsaw, opacitybacktitle=0.6, rightrule=.15mm, toprule=.15mm, arc=.35mm, left=2mm, toptitle=1mm, title=\textcolor{quarto-callout-tip-color}{\faLightbulb}\hspace{0.5em}{Solución}, titlerule=0mm, opacityback=0, colback=white, bottomtitle=1mm, bottomrule=.15mm, colframe=quarto-callout-tip-color-frame, leftrule=.75mm, breakable, coltitle=black, colbacktitle=quarto-callout-tip-color!10!white]

Sea \(A\sim B(500, p_A)\) la variable aleatoria que mide el número de
personas que votarían al partido \(A\) y \(B\sim B(1000, p_B)\) la
variable aleatoria que mide el número de personas que votarían al
partido \(B\) en una muestra de 1000 personas.

Usando el intervalo de confianza de la normal se tiene que el porcentaje
de voto al partido \(A\) está entre el \(32.04\)\% y el \(37.96\)\% y el
porcentaje de voto al partido \(B\) está entre el \(35.98\)\% y el
\(42.02\) con una confianza del \(95\)\%.

\end{tcolorbox}

\begin{exercise}[]\protect\hypertarget{exr-intervalo-confianza-ventas-fármaco}{}\label{exr-intervalo-confianza-ventas-fármaco}

Para ver si una campaña de publicidad sobre un fármaco ha influido en
sus ventas, se tomó una muestra de \(8\) farmacias y se midió el número
de fármacos vendidos durante un mes, antes y después de la campaña,
obteniéndose los siguientes resultados:

\[
\begin{array}{lcccccccc}
\hline
\mbox{Antes} & 147 & 163 & 121 & 205 & 132 & 190 & 176 & 147  \\
\mbox{Después} & 150 & 171 & 132 & 208 & 141 & 184 & 182 & 145  \\
\hline
\end{array}
\]

Obtener la variable diferencia y construir un intervalo de confianza
para la media de la diferencia con un nivel de significación \(0.05\).
¿Existen pruebas suficientes para afirmar con un \(95\)\% de confianza
que la campaña de publicidad ha aumentado las ventas?

\end{exercise}

\begin{tcolorbox}[enhanced jigsaw, opacitybacktitle=0.6, rightrule=.15mm, toprule=.15mm, arc=.35mm, left=2mm, toptitle=1mm, title=\textcolor{quarto-callout-tip-color}{\faLightbulb}\hspace{0.5em}{Solución}, titlerule=0mm, opacityback=0, colback=white, bottomtitle=1mm, bottomrule=.15mm, colframe=quarto-callout-tip-color-frame, leftrule=.75mm, breakable, coltitle=black, colbacktitle=quarto-callout-tip-color!10!white]

Sea \(X\) la variable aleatoria que mide la diferencia entre el número
de fármacos vendidos en una farmacia en un mes antes y después de la
campaña de publicidad.

Usando el intervalo de confianza de la t de Student se tiene que
\(\mu \in (-1.75, 21.75)\) con una confianza del \(95\)\%.

\end{tcolorbox}

\begin{exercise}[]\protect\hypertarget{exr-intervalo-confianza-comparación-medias-ventas}{}\label{exr-intervalo-confianza-comparación-medias-ventas}

Se ha realizado un estudio para comparar los ingresos medios de las
personas de dos ciudades. Para ello, se ha tomado una muestra de \(100\)
personas en una ciudad y \(120\) en la otra. En la primera ciudad se ha
observado una media de \(1630\) € mensuales y una cuasidesviación típica
de \(150\) €, mientras que en la segunda ciudad se ha observado una
media de \(1780\)€ y una cuasidesviación típica de \(160\) €. Calcular
el intervalo de confianza del \(95\)\% para la diferencia de medias de
ingresos mensuales entre las dos ciudades.

\end{exercise}

\begin{tcolorbox}[enhanced jigsaw, opacitybacktitle=0.6, rightrule=.15mm, toprule=.15mm, arc=.35mm, left=2mm, toptitle=1mm, title=\textcolor{quarto-callout-tip-color}{\faLightbulb}\hspace{0.5em}{Solución}, titlerule=0mm, opacityback=0, colback=white, bottomtitle=1mm, bottomrule=.15mm, colframe=quarto-callout-tip-color-frame, leftrule=.75mm, breakable, coltitle=black, colbacktitle=quarto-callout-tip-color!10!white]

Sea \(X_1\sim N(\mu_1, \sigma_1)\) la variable aleatoria que mide los
ingresos mensuales de las personas en la primera ciudad y
\(X_2\sim N(\mu_2, \sigma_2)\) la variable aleatoria que mide los
ingresos mensuales de las personas en la segunda ciudad.

El intervalo de confianza del \%95\% para el cociente de varianzas es
\(\frac{\sigma_1 2}{\sigma_2^2}\in (0.6273, 1.2492)\) por lo que podemos
asumir varianzas iguales, y el intervalo de confianza para para la
diferencia de medias es \(\mu_1-\mu_2\in (-191.20, -108.80)\) con una
confianza del \(95\)\%.

\end{tcolorbox}

\begin{exercise}[]\protect\hypertarget{exr-intervalo-confianza-comparación-proporciones-emisiones}{}\label{exr-intervalo-confianza-comparación-proporciones-emisiones}

La siguiente tabla muestra el porcentaje de industrias españolas y
europeas según el consumo de energía primaria de las mismas durante el
año 2002 (se estudiaron 1000 industrias españolas y 1000 del resto de
europa).

\[
\begin{array}{lcc}
\hline
\mbox{Fuente energética}  &  \mbox{España}  & \mbox{Resto de Europa} \\
\mbox{Petróleo}            & 52.2\% &    40.4\%     \\
\mbox{Carbón}              & 15.2\% &    14.8\%     \\
\mbox{Nuclear}             & 13.0\% &    15.2\%     \\
\mbox{Gas}                 & 12.8\% &    23.5\%     \\
\mbox{Energías Renovables} & 6.5\%  &     6.1\%     \\
\hline
\end{array}
\]

Estudiar, según los intervalos de confianza para diferencia de
proporciones, para qué energías el porcentaje de industrias de España es
significativamente diferente del resto de Europa.

\end{exercise}

\begin{tcolorbox}[enhanced jigsaw, opacitybacktitle=0.6, rightrule=.15mm, toprule=.15mm, arc=.35mm, left=2mm, toptitle=1mm, title=\textcolor{quarto-callout-tip-color}{\faLightbulb}\hspace{0.5em}{Solución}, titlerule=0mm, opacityback=0, colback=white, bottomtitle=1mm, bottomrule=.15mm, colframe=quarto-callout-tip-color-frame, leftrule=.75mm, breakable, coltitle=black, colbacktitle=quarto-callout-tip-color!10!white]

Los intervalos de confianza del \(95\)\% para la diferencia de
proporciones de industrias que usan las diferentes fuentes de energía en
España y en el resto de Europa son:

\begin{itemize}
\tightlist
\item
  Petróleo: \((0.0747, 0.1613)\). Hay diferencias significativas.
\item
  Carbón: \((-0.0273, 0.0353)\). No hay diferencias significativas.
\item
  Nuclear: \((-0.0525, 0.0085)\). No hay diferencias significativas.
\item
  Gas: \((-0.1405, -0.0735)\). Hay diferencias significativas.
\item
  Energías Renovables: \((-0.0173, 0.0253)\). No hay diferencias
  significativas.
\end{itemize}

\end{tcolorbox}

\bookmarksetup{startatroot}

\chapter{Contrastes de hipótesis
paramétricos}\label{contrastes-de-hipuxf3tesis-paramuxe9tricos}

\begin{exercise}[]\protect\hypertarget{exr-contraste-media-consumo-azucar}{}\label{exr-contraste-media-consumo-azucar}

Sabiendo que el año pasado el consumo per cápita de azúcar en España fue
de 4.8 kg y que este consumo sigue una distribución normal, hemos
seleccionado aleatoriamente a 20 españoles obteniendo una media muestral
de 5 kg y una cuasidesviación típica muestral de 0.4 kg. Contrastar la
hipótesis de que el consumo de azúcar per cápita de este año en España
no ha variado utilizando un nivel de significación del 10\% en cada uno
de los casos siguientes.

\begin{enumerate}
\def\labelenumi{\alph{enumi}.}
\item
  Suponiendo que la alternativa es que el consumo de azúcar per cápita
  sea distinto.
\item
  Suponiendo que la alternativa es que el consumo de azúcar per cápita
  sea mayor.
\end{enumerate}

\end{exercise}

\begin{exercise}[]\protect\hypertarget{exr-contraste-proporcion-asistencia-clase}{}\label{exr-contraste-proporcion-asistencia-clase}

En una clase de Estadística se ha comprobado que el 20\% del alumnado
falta a clase. Para disminuir esta preocupante cifra, los profesores han
incorporado un sistema de evaluación continua que tendrá en cuenta las
notas de clase de los alumnos en la nota final. Contraste con un nivel
de significación del 5\% que la incorporación de este método no es
efectiva, es decir, el absentismo antes y después de la evaluación
continua es el mismo, sabiendo que el porcentaje medio de no asistencia
en 50 días tomados al azar ha sido del 17\%.

\end{exercise}

\begin{exercise}[]\protect\hypertarget{exr-contraste-media-detergente}{}\label{exr-contraste-media-detergente}

Una empresa fabricante de detergente afirma que el contenido de cada
paquete de detergente sigue una distribución normal de media 2150 g,
pero una Asociación de Consumidores no está conforme con esta
afirmación, por lo que realiza un estudio consistente en obtener una
muestra aleatoria simple de 121 paquetes de detergente, obteniendo un
contenido medio muestral de 2070 g y una cuasidesviación típica muestral
de 130 grs. Contraste esta hipótesis con un nivel de significación del
5\%.

\end{exercise}

\begin{exercise}[]\protect\hypertarget{exr-contraste-proporcion-aprobados}{}\label{exr-contraste-proporcion-aprobados}

El número de aprobados en una asignatura de un determinado curso ha sido
del 64\%. En uno de los grupos de ese curso se han presentado al examen
40 alumnos de los que 31 aprobaron. ¿Puede afirmarse con un nivel de
significación del 5\% que los alumnos de dicho grupo han obtenido
mejores calificaciones que el resto de los alumnos del curso?

\end{exercise}

\begin{exercise}[]\protect\hypertarget{exr-contraste-media-consumo}{}\label{exr-contraste-media-consumo}

Se sabe que el consumo anual de helado correspondiente a cada español
sigue una distribución normal y que el año pasado el consumo medio fue
de 20 kg. Queremos contrastar si este año se va a mantener el consumo
medio de helado que el año pasado, y para comprobarlo se efectúa una
muestra aleatoria de 22 españoles, obteniéndose los siguientes
resultados:

15, 18, 24, 31, 22, 12, 6, 35, 42, 2, 16, 25, 20, 10, 17, 19, 14, 30,
14, 23, 15, 19.

Realizar el contraste con un nivel de significación de 0.10.

\end{exercise}

\begin{exercise}[]\protect\hypertarget{exr-contraste-varianza-telefonica}{}\label{exr-contraste-varianza-telefonica}

Telefónica ha constatado que el consumo de datos de los clientes que han
contratado el paquete Fusión (fibra simétrica, TV, teléfono fijo y
móvil. sigue una distribución normal cuya dispersión viene determinada
por \(\sigma\)=3 Gb. Sin embargo, tras la incorporación de Netflix a la
oferta de Movistar TV se ha observado que la dispersión habría podido
cambiar, por lo que se ha llevado a cabo un muestreo con 15 clientes en
el que la cuasidesviación típica es igual a 3.5 Gb.

Determine si efectivamente la varianza en el consumo de datos de los
clientes Fusión ha cambiado tras la incorporación de Netflix y
especifique la región crítica óptima utilizando un nivel de
significación del 2\%.

\end{exercise}

\begin{exercise}[]\protect\hypertarget{exr-contraste-media-peso}{}\label{exr-contraste-media-peso}

La OMS está preocupada por el incremento de la obesidad entre los niños
de entre 11 y 14 años. La variable peso de los niños se distribuye según
una normal. El equipo formula la hipótesis de que el peso medio de los
niños de entre 11 y 14 años es de 46.5 kg. Seleccionada una muestra
aleatoria de 40 niños, se obtiene que la media muestral es 49 kg y la
cuasidesviación típica muestral vale 7 kg.

Contrastar con un nivel de significación del 10\% que el peso medio de
los niños de entre 11 y 14 años sea 46.5 kg, frente a que sea mayor.

\end{exercise}

\begin{exercise}[]\protect\hypertarget{exr-contraste-proporcion-consumo-producto}{}\label{exr-contraste-proporcion-consumo-producto}

Al lanzar un nuevo producto al mercado, el fabricante duda entre que sea
adquirido por el 20\% de la población o que sea adquirido por el 30\%.
Seleccionada al azar una muestra de 400 posibles compradores del
producto se obtiene que la demandaría el 22\%. La regla de decisión que
utilizaremos es aceptar la hipótesis nula si adquieren el producto menos
del 25\% de los consultados.

\begin{enumerate}
\def\labelenumi{\alph{enumi}.}
\item
  ¿Se puede aceptar la hipótesis nula?
\item
  Calcular el nivel de significación del contraste.
\item
  Obtener la potencia del contraste.
\end{enumerate}

\end{exercise}

\begin{exercise}[]\protect\hypertarget{exr-contraste-media-consumo-electrico}{}\label{exr-contraste-media-consumo-electrico}

Según los datos publicados por REE (Red Eléctrica Española. el consumo
medio diario de electricidad en los hogares españoles es de 9,55
watios/hora. La compañía comercializadora de electricidad Holaluz sabe
que el citado consumo sigue una distribución normal, pero debe
determinar cuál será el consumo medio en el que basará su estrategia de
cara a 2021, ya que el aumento del teletrabajo podría provocar que el
referido consumo medio varíe en el nuevo ejercicio y se incremente a
12,50 watios/hora, por lo que ha llevado a cabo un muestra aleatoria con
25 clientes en el que la media muestral ha resultado ser igual a un
consumo de 12,17 watios/hora mientras que la cuasivarianza es igual a 9.

\begin{enumerate}
\def\labelenumi{\alph{enumi}.}
\item
  ¿Con que hipótesis debería trabajar Holaluz en 2021 si se considera un
  nivel de significación del 1\%?
\item
  ¿Cuál es la potencia del contraste?
\end{enumerate}

\end{exercise}

\begin{exercise}[]\protect\hypertarget{exr-contraste-proporcion-vacunados-covid}{}\label{exr-contraste-proporcion-vacunados-covid}

En la campaña de vacunación que recientemente ha comenzado en España
para hacer frente a la pandemia provocada por la COVID, se ha estimado
que un 24\% de la población se negará a recibir la vacuna. No obstante,
se cree que a medida que el proceso de vacunación avance, este
porcentaje podría variar y reducirse al 19\%, por lo que se ha decidido
realizar un muestreo aleatorio entre la población española a los pocos
días de comenzar la campaña de vacunación en el que se ha preguntado a
200 personas si se vacunarán cuando sean citados por las autoridades
sanitarias, habiendo respondido 42 personas que no acudirán a vacunarse
aduciendo diversas razones.

\begin{enumerate}
\def\labelenumi{\alph{enumi}.}
\item
  Aplicando un nivel de significación del 2\% ¿se podría afirmar que el
  porcentaje de personas que se niegan a vacunarse ha cambiado?
\item
  ¿Cuál es la potencia del contraste?
\end{enumerate}

\end{exercise}

\begin{exercise}[]\protect\hypertarget{exr-contraste-varianza-principio-activo}{}\label{exr-contraste-varianza-principio-activo}

Se sabe que el contenido de principio activo de los medicamentos
producidos por una máquina sigue una distribución normal con varianza
0.2 mg/ml. Tras una revisión de mantenimiento y calibración de la
máquina se cree que la variabilidad del contenido de principio activo ha
disminuido 0.1 mg/ml. Para contrastarlo, se ha tomado una muestra de 10
medicamentos en los que se ha observado una varianza de 0.12 mg/ml. ¿Se
puede aceptar la hipótesis de que la variabilidad ha disminuido 0.1
mg/ml?

\end{exercise}



\end{document}
